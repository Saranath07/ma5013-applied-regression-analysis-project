%%%%%%%%%%%%%%%%%%%%%%%%%%%%%%%%%%%%%%%%%
% Beamer Presentation (Detailed Walkthrough - CORRECTED)
%
% This version removes the faulty \newcommand definitions
% and will compile correctly.
%%%%%%%%%%%%%%%%%%%%%%%%%%%%%%%%%%%%%%%%%

\documentclass[aspectratio=169]{beamer}

\mode<presentation>
{
  % Theme choice
  \usetheme{Singapore} % A clean theme that handles content well
  \usecolortheme{default}
  \usefonttheme{default}
  
  % Template settings
  \setbeamertemplate{navigation symbols}{}
  \setbeamertemplate{caption}[numbered]
  \setbeamertemplate{footline}{%
    \leavevmode%
    \hbox{%
    \begin{beamercolorbox}[wd=.5\paperwidth,ht=2.25ex,dp=1ex,center]{author in head/foot}%
      \usebeamerfont{author in head/foot}\insertshortauthor
    \end{beamercolorbox}%
    \begin{beamercolorbox}[wd=.5\paperwidth,ht=2.25ex,dp=1ex,right]{date in head/foot}%
      \usebeamerfont{date in head/foot}\insertshortdate{}\hspace*{2em}
      \insertframenumber{} / \inserttotalframenumber\hspace*{2ex}
    \end{beamercolorbox}}%
    \vskip0pt%
  }
} 

\usepackage[utf8]{inputenc}
\usepackage{amsmath}
\usepackage{amsfonts}
\usepackage{amssymb}
\usepackage{graphicx}
\usepackage{booktabs} % For professional tables
\usepackage{verbatim} % For verbatim text like code output


%------------------------------------------------------------
% Title Page
%------------------------------------------------------------
\title[Spotify Hit Song Formula]{The Hit Song Formula: A Methodological Journey}
\subtitle{An Applied Regression Analysis of Spotify Song Popularity}
\author{Saranath P \and S Shriprasad}
\institute[IIT Madras]{Indian Institute of Technology Madras \\ \medskip MA 5013: Applied Regression Analysis}
\date{Fall 2025}

%------------------------------------------------------------
\begin{document}
%------------------------------------------------------------
\begin{frame}
  \titlepage
\end{frame}
%------------------------------------------------------------
\begin{frame}{Introduction: The Quest for a Hit Song}
    \begin{columns}[T]
        \begin{column}{0.6\textwidth}
            \begin{block}{Motivation}
                The music industry perpetually seeks the "recipe" for a hit song. While art is subjective, can we find statistical patterns in a song's audio features that correlate with its success?
            \end{block}
            
            \begin{block}{Objective \& Response}
                \textbf{Objective:} To build a valid regression model predicting a song's popularity on Spotify.
                \vfill
                \textbf{Response Variable:} $y$ (popularity), a continuous score from 0 to 100 from the Spotify API.
            \end{block}
        \end{column}
        
        \begin{column}{0.4\textwidth}
            \begin{figure}
                \includegraphics[width=0.8\textwidth]{spotify-logo.png}
                \caption{Dataset from Kaggle's "Ultimate Spotify Tracks Database". We analyzed a random sample of 10,000 songs.}
            \end{figure}
        \end{column}
    \end{columns}
\end{frame}
%------------------------------------------------------------
\begin{frame}{Our Guiding Research Questions}
    
    Our investigation was structured around four key hypotheses:

    \begin{enumerate}
        \item \textbf{The "Big Five":} Which core audio features (\textit{danceability, energy, valence, etc.}) have the most significant impact on popularity?

        \item \textbf{The "Goldilocks Zone" Hypothesis:} Do non-linear "sweet spots" exist? Are songs with moderate \textit{tempo} and \textit{duration} more popular than extremes?

        \item \textbf{The "Sad Banger" Phenomenon:} Does the combination of high energy and low valence (sadness) lead to disproportionately popular songs? (Interaction: \textit{energy} $\times$ \textit{valence})

        \item \textbf{The "Acoustic Amplification" Effect:} Does a song's acoustic nature change the impact of its emotional tone on popularity? (Interaction: \textit{acousticness} $\times$ \textit{valence})
    \end{enumerate}
\end{frame}
%------------------------------------------------------------
\begin{frame}{Exploratory Data Analysis (EDA): Initial Insights}
    \begin{columns}[T]
        \begin{column}{0.5\textwidth}
            \begin{figure}
                \includegraphics[width=\textwidth, height=0.55\textheight, keepaspectratio]{plot_eda_popularity_dist.png.png}
                \caption{Popularity distribution: bell-shaped with spike near zero.}
            \end{figure}
        \end{column}
        \begin{column}{0.5\textwidth}
            \begin{figure}
                \includegraphics[width=\textwidth, height=0.55\textheight, keepaspectratio]{plot_eda_correlation_matrix.png}
                \caption{Correlation matrix shows multicollinearity.}
            \end{figure}
        \end{column}
    \end{columns}
\end{frame}
%------------------------------------------------------------
\section[The Naive Model and Its Failures]{The Naive Model \& Its Failures}
%------------------------------------------------------------
\begin{frame}{Part 1: The Baseline OLS Model}
    
    Our first step was to fit a standard Ordinary Least Squares (OLS) model using the "Big Five" predictors.
    
    \begin{block}{Initial Model Specification}
    $y = \beta_0 + \beta_1 x_1 + \beta_2 x_2 + \beta_3 x_3 + \beta_4 x_4 + \beta_5 x_5 + \epsilon$
    \end{block}
    
    \begin{block}{Variable Definitions}
    \begin{columns}[T]
        \begin{column}{0.5\textwidth}
            \begin{itemize}
                \item $y$ = popularity
                \item $x_1$ = danceability
                \item $x_2$ = energy
            \end{itemize}
        \end{column}
        \begin{column}{0.5\textwidth}
            \begin{itemize}
                \item $x_3$ = valence
                \item $x_4$ = acousticness
                \item $x_5$ = instrumentalness
            \end{itemize}
        \end{column}
    \end{columns}
    \end{block}
    
    This model serves as our naive baseline, which we must rigorously validate before accepting its results.
    
\end{frame}
%------------------------------------------------------------
\begin{frame}[fragile]{Baseline OLS Model: Summary Table}
    \frametitle{Baseline OLS Model: Summary Table}
    \begin{block}{OLS Regression Results}
    \vspace{0.5em}
    {\tiny
    \begin{tabular}{p{3cm}rrrrrr}
    \toprule
    \multicolumn{7}{c}{\textbf{Model Summary}} \\
    \midrule
    \multicolumn{2}{l}{Dep. Variable: $y$ (popularity)} & \multicolumn{2}{l}{R-squared: 0.184} & \multicolumn{2}{l}{F-statistic: 450.5} & \\
    \multicolumn{2}{l}{Model: OLS} & \multicolumn{2}{l}{Adj. R-squared: 0.184} & \multicolumn{2}{l}{Prob (F-statistic): 0.00} & \\
    \multicolumn{2}{l}{Method: Least Squares} & \multicolumn{2}{l}{No. Observations: 10000} & \multicolumn{2}{l}{AIC: 8.457e+04} & \\
    \midrule
    \multicolumn{7}{c}{\textbf{Coefficients}} \\
    \midrule
    \textbf{Variable} & \textbf{coef} & \textbf{std err} & \textbf{t} & \textbf{$P>|t|$} & \textbf{[0.025} & \textbf{0.975]} \\
    \midrule
    $\beta_0$ (const) & 43.72 & 0.986 & 44.34 & 0.000 & 41.78 & 45.65 \\
    $\beta_1$ ($x_1$: danceability) & 20.73 & 1.124 & 18.45 & 0.000 & 18.53 & 22.94 \\
    $\beta_2$ ($x_2$: energy) & -1.64 & 0.978 & -1.68 & 0.093 & -3.56 & 0.28 \\
    $\beta_3$ ($x_3$: valence) & -12.85 & 0.805 & -15.97 & 0.000 & -14.43 & -11.28 \\
    $\beta_4$ ($x_4$: acousticness) & -18.27 & 0.697 & -26.21 & 0.000 & -19.63 & -16.90 \\
    $\beta_5$ ($x_5$: instrumentalness) & -4.40 & 0.617 & -7.12 & 0.000 & -5.61 & -3.19 \\
    \bottomrule
    \end{tabular}
    }
    \end{block}
    \begin{alertblock}{Initial Interpretation}
    The model appears significant (F-stat), but `energy` is not. `danceability` has a strong positive effect, while others are negative. \textbf{But are these results valid?}
    \end{alertblock}
\end{frame}
%------------------------------------------------------------
\begin{frame}{Diagnostic 1: Multicollinearity Detection}
    \begin{columns}[T]
        \begin{column}{0.4\textwidth}
            \begin{block}{Variance Inflation Factor (VIF)}
                We test for multicollinearity using VIF. A VIF $>$ 5 indicates a problem.
            \end{block}
            \begin{alertblock}{Problem Detected}
            \textbf{Three variables} exceed threshold:
            \begin{itemize}
                \item $x_1$: \textbf{10.17}
                \item $x_2$: \textbf{6.61}
                \item $x_3$: \textbf{6.50}
            \end{itemize}
            \end{alertblock}
        \end{column}
        \begin{column}{0.6\textwidth}
            \begin{block}{VIF Scores Table}
            \vspace{0.5em}
            {\footnotesize
            \begin{tabular}{lr}
            \toprule
            Variable & VIF \\
            \midrule
            $x_1$ (danceability) & \textcolor{red}{\textbf{10.17}} \\
            $x_2$ (energy) & \textcolor{red}{\textbf{6.61}} \\
            $x_3$ (valence) & \textcolor{red}{\textbf{6.50}} \\
            $x_4$ (acousticness) & 2.14 \\
            $x_5$ (instrumentalness) & 1.39 \\
            \bottomrule
            \end{tabular}
            }
            \end{block}
        \end{column}
    \end{columns}
    \begin{beamercolorbox}[rounded=true,shadow=true]{block body}
    \textbf{Impact:} Coefficients and p-values unreliable.
    \end{beamercolorbox}
\end{frame}
%------------------------------------------------------------
\begin{frame}{Multicollinearity Treatment Strategy}
    \begin{alertblock}{Treatment Approach}
    Remove highest VIF variable ($x_1$: danceability) first.
    \end{alertblock}
    
    \begin{block}{Why Remove Danceability?}
    \begin{itemize}
        \item Highest VIF score (10.17)
        \item Sequential removal approach
        \item Maintains interpretability
    \end{itemize}
    \end{block}
    
    \begin{block}{Expected Outcome}
    After removing $x_1$:
    \begin{itemize}
        \item VIF scores decrease
        \item Coefficients more stable
        \item Tests more reliable
    \end{itemize}
    \end{block}
\end{frame}
%------------------------------------------------------------
\begin{frame}{Diagnostic 2: Normality of Residuals}
    \begin{columns}[T]
        \begin{column}{0.5\textwidth}
            \begin{block}{The Assumption}
                OLS regression assumes that the model's errors (residuals) are normally distributed. We check this with a Q-Q plot.
            \end{block}
            \begin{alertblock}{Diagnosis: SEVERE VIOLATION}
            The residuals deviate drastically from the theoretical normal line.
            \vfill
            This heavy-tailed 'S' curve invalidates all p-values, confidence intervals, and hypothesis tests from the OLS summary.
            \end{alertblock}
            \textbf{Conclusion: The OLS model is statistically invalid.}
        \end{column}
        \begin{column}{0.5\textwidth}
            \begin{figure}
                \includegraphics[width=0.7\textwidth]{plot_ols_qq.png}
                \caption{Q-Q Plot for the Baseline OLS Model}
            \end{figure}
        \end{column}
    \end{columns}
\end{frame}
%------------------------------------------------------------
\section{The Diagnostic Journey}
%------------------------------------------------------------
\begin{frame}{Identifying the Root Cause: Skewness or Outliers?}
    What is causing the non-normal residuals?
    
    \begin{block}{Hypothesis: The response variable is skewed}
        If skewed, Box-Cox transformation should normalize residuals.
    \end{block}
    
    \begin{block}{Box-Cox Results}
        Optimal lambda = \textbf{1.0252} $\approx$ 1 (no transformation needed).
        
        This suggests transformation will not fix normality.
    \end{block}
    
    \begin{alertblock}{Conclusion}
    Issue is likely \textbf{not simple skewness}. Need to investigate outliers.
    \end{alertblock}
\end{frame}
%------------------------------------------------------------
\begin{frame}{Box-Cox Results: Visual Evidence}
    \begin{columns}[T]
        \begin{column}{0.5\textwidth}
            \begin{figure}
                \includegraphics[width=0.7\textwidth]{plot_boxcox_qq.png}
                \caption{Q-Q Plot after Box-Cox transformation}
            \end{figure}
        \end{column}
        \begin{column}{0.5\textwidth}
            \begin{block}{Key Observations}
                \begin{itemize}
                    \item Heavy tails persist after transformation
                    \item Points deviate from normal line
                    \item No improvement in normality
                \end{itemize}
            \end{block}
            
            \begin{alertblock}{Crucial Insight}
            The problem is not simple skewness. The heavy tails point to \textbf{influential outliers}.
            \end{alertblock}
            
            \begin{block}{Next Step}
            Identify influential observations using Cook's Distance.
            \end{block}
        \end{column}
    \end{columns}
\end{frame}
%------------------------------------------------------------
\begin{frame}{Confirming the Root Cause: Influential Outliers}
    \begin{columns}[T]
        \begin{column}{0.5\textwidth}
            \begin{block}{Cook's Distance}
                Cook's Distance measures how much the entire regression model changes when a single observation is removed. High values indicate influential points.
            \end{block}
            \begin{alertblock}{Diagnosis: Severe Problem Detected}
            The plot reveals numerous points with high influence.
            \vfill
            We identified \textbf{455 influential outliers} (where Cook's $D > 4/n$). These points are pulling the regression line and distorting the residuals for all other points.
            \end{alertblock}
        \end{column}
        \begin{column}{0.5\textwidth}
            \begin{figure}
                \includegraphics[width=\textwidth]{plot_cooks_distance.png}
                \caption{Cook's Distance Plot for Naive OLS Model}
            \end{figure}
        \end{column}
    \end{columns}
\end{frame}
%------------------------------------------------------------
\section[The Solution: Robust and Stable Modeling]{The Solution: Robust \& Stable Modeling}
%------------------------------------------------------------
\begin{frame}{The Right Tool: Robust Linear Models (RLM)}
    Since the problem is outliers, we need a method designed to handle them.
    
    \begin{columns}[T]
        \begin{column}{0.5\textwidth}
            \begin{block}{Robust Linear Model (RLM)}
                RLM works by an iterative process (IRLS) that systematically down-weights the influence of observations identified as outliers.
                \vspace{1em}
                
                This forces the model to fit the bulk of the data, not the extremes.
            \end{block}
            
            \begin{alertblock}{Key Advantage}
            RLM automatically identifies and reduces the impact of problematic observations.
            \end{alertblock}
        \end{column}
        \begin{column}{0.5\textwidth}
            \begin{figure}
                \includegraphics[width=\textwidth]{plot_rlm_weights.png}
                \caption{Proof RLM is working: Points with high Cook's Distance are assigned low weights}
            \end{figure}
        \end{column}
    \end{columns}
\end{frame}
%------------------------------------------------------------
\begin{frame}[fragile]{Applying RLM to the Baseline Model}
    \frametitle{Applying RLM to the Baseline Model}
    \begin{block}{Robust Regression Results (with all "Big Five")}
    \vspace{0.5em}
    {\tiny
    \begin{tabular}{lrrrrrr}
    \toprule
    \multicolumn{7}{c}{\textbf{RLM Model Summary}} \\
    \midrule
    \multicolumn{2}{l}{Dep. Variable: $y$ (popularity)} & \multicolumn{2}{l}{No. Observations: 10000} & \multicolumn{2}{l}{Df Model: 5} & \\
    \multicolumn{2}{l}{Model: RLM} & \multicolumn{2}{l}{Df Residuals: 9994} & \multicolumn{2}{l}{Method: IRLS} & \\
    \midrule
    \multicolumn{7}{c}{\textbf{Coefficients}} \\
    \midrule
    \textbf{Variable} & \textbf{coef} & \textbf{std err} & \textbf{z} & \textbf{$P>|z|$} & \textbf{[0.025} & \textbf{0.975]} \\
    \midrule
    $\beta_0$ (const) & 44.59 & 1.031 & 43.23 & 0.000 & 42.57 & 46.61 \\
    $\beta_1$ ($x_1$: danceability) & 22.20 & 1.175 & 18.88 & 0.000 & 19.89 & 24.50 \\
    $\beta_2$ ($x_2$: energy) & -4.07 & 1.023 & -3.98 & 0.000 & -6.08 & -2.07 \\
    $\beta_3$ ($x_3$: valence) & -11.89 & 0.842 & -14.12 & 0.000 & -13.54 & -10.24 \\
    $\beta_4$ ($x_4$: acousticness) & -18.98 & 0.729 & -26.03 & 0.000 & -20.41 & -17.55 \\
    $\beta_5$ ($x_5$: instrumentalness) & -4.64 & 0.646 & -7.18 & 0.000 & -5.91 & -3.37 \\
    \bottomrule
    \end{tabular}
    }
    \end{block}
    \begin{alertblock}{A Key Insight Emerges}
    In the OLS model, $x_2$ (energy) was insignificant (p=0.093). In the robust model, \textbf{$x_2$ (energy) is now highly significant (p $<$ 0.001)}. The outliers were masking its true negative relationship with popularity!
    \end{alertblock}
\end{frame}
%------------------------------------------------------------
\begin{frame}[fragile]{Creating a Stable Baseline: Removing Multicollinearity}
    \frametitle{Creating a Stable Baseline: Removing Multicollinearity}
    Now that we have a robust method, we can safely remove the collinear predictor (`danceability`) identified earlier.
    \begin{columns}[T]
        \begin{column}{0.65\textwidth}
            \begin{block}{Final Stable Baseline RLM Results}
            \vspace{0.5em}
            {\tiny
            \begin{tabular}{lrrrrrr}
            \toprule
            \multicolumn{7}{c}{\textbf{Final Baseline Model}} \\
            \midrule
            \multicolumn{2}{l}{Model: RLM} & \multicolumn{2}{l}{No. Obs: 10000} & \multicolumn{2}{l}{Df Model: 4} & \\
            \midrule
            \textbf{Variable} & \textbf{coef} & \textbf{std err} & \textbf{z} & \textbf{$P>|z|$} & \textbf{[0.025} & \textbf{0.975]} \\
            \midrule
            $\beta_0$ (const) & 56.22 & 0.847 & 66.41 & 0.000 & 54.56 & 57.88 \\
            $\beta_2$ ($x_2$: energy) & -5.97 & 1.039 & -5.75 & 0.000 & -8.01 & -3.94 \\
            $\beta_3$ ($x_3$: valence) & -4.83 & 0.761 & -6.34 & 0.000 & -6.32 & -3.34 \\
            $\beta_4$ ($x_4$: acousticness) & -21.79 & 0.729 & -29.87 & 0.000 & -23.22 & -20.36 \\
            $\beta_5$ ($x_5$: instrumentalness) & -7.46 & 0.642 & -11.61 & 0.000 & -8.72 & -6.20 \\
            \bottomrule
            \end{tabular}
            }
            \end{block}
        \end{column}
        \begin{column}{0.35\textwidth}
            \begin{block}{Final VIF Scores}
            \vspace{0.5em}
            {\footnotesize
            \begin{tabular}{lr}
            \toprule
            Variable & VIF \\
            \midrule
            $x_3$ (valence) & \textcolor{blue}{\textbf{4.82}} \\
            $x_2$ (energy) & \textcolor{blue}{\textbf{4.44}} \\
            $x_4$ (acousticness) & 1.75 \\
            $x_5$ (instrumentalness) & 1.39 \\
            \bottomrule
            \end{tabular}
            }
            \end{block}
            
            \begin{alertblock}{Success!}
            All VIF scores now $<$ 5. Multicollinearity resolved.
            \end{alertblock}
        \end{column}
    \end{columns} 
    \vspace{1em}
    \begin{beamercolorbox}[rounded=true,shadow=true]{block body}
    We now have a \textbf{doubly-corrected baseline model}: it is robust to outliers AND free of severe multicollinearity. This is our foundation.
    \end{beamercolorbox}
\end{frame}
%------------------------------------------------------------
\section[The Final Model and Results]{The Final Model \& Results}
%------------------------------------------------------------
\begin{frame}[fragile]{The Full Model: Testing All Hypotheses}
    \frametitle{The Full Model: Testing All Hypotheses}
    We now build the full model on our stable foundation, adding quadratic and interaction terms to test RQ2, RQ3, and RQ4.
    
    \begin{block}{Full Model RLM Results}
    \vspace{0.5em}
    {\tiny
    \begin{tabular}{lrrrrrr}
    \toprule
    \textbf{Variable} & \textbf{coef} & \textbf{std err} & \textbf{z} & \textbf{$P>|z|$} & \textbf{[0.025} & \textbf{0.975]} \\
    \midrule
    $\beta_0$ (const) & 60.09 & 1.526 & 39.39 & 0.000 & 57.11 & 63.09 \\
    $\beta_2$ ($x_2$: energy) & -8.37 & 1.910 & -4.38 & 0.000 & -12.12 & -4.63 \\
    $\beta_3$ ($x_3$: valence) & -11.00 & 3.076 & -3.58 & 0.000 & -17.02 & -4.97 \\
    $\beta_4$ ($x_4$: acousticness) & -25.33 & 1.416 & -17.88 & 0.000 & -28.10 & -22.55 \\
    $\beta_5$ ($x_5$: instrumentalness) & -6.96 & 0.651 & -10.69 & 0.000 & -8.24 & -5.68 \\
    tempo\_c & 0.012 & 0.006 & 1.94 & 0.053 & -0.000 & 0.025 \\
    tempo\_c\_sq & -0.001 & 0.000 & -7.61 & 0.000 & -0.002 & -0.001 \\
    duration\_s\_c & 0.009 & 0.002 & 4.39 & 0.000 & 0.005 & 0.013 \\
    duration\_s\_c\_sq & -1.75e-05 & 1.52e-06 & -11.53 & 0.000 & -2.05e-05 & -1.46e-05 \\
    energy\_valence\_interact & 4.95 & 3.802 & 1.30 & 0.193 & -2.50 & 12.40 \\
    acoustic\_valence\_interact & 10.49 & 2.801 & 3.75 & 0.000 & 5.00 & 15.98 \\
    \bottomrule
    \end{tabular}
    }
    \end{block}
\end{frame}
%------------------------------------------------------------
\begin{frame}{Full Model Analysis}
    \begin{alertblock}{Observation}
    Some predictors not significant (energy×valence p=0.193, tempo p=0.053). Need backward elimination.
    \end{alertblock}
    
    \begin{block}{Key Findings}
    \begin{itemize}
        \item Core features ($x_2$-$x_5$) highly significant
        \item Quadratic terms confirm non-linear effects
        \item acoustic×valence significant (p $<$ 0.001)
        \item energy×valence not significant (p = 0.193)
    \end{itemize}
    \end{block}
    
    \begin{alertblock}{Next Step}
    Remove non-significant predictors for parsimonious model.
    \end{alertblock}
\end{frame}
%------------------------------------------------------------
\begin{frame}[fragile]{The Final Parsimonious Model}
    \frametitle{The Final Parsimonious Model}
    After removing insignificant predictors (\texttt{energy\_valence\_interact} and \texttt{tempo\_c}), we arrive at our final model where every variable is statistically significant (p $<$ 0.05).
    
    \begin{block}{Final Parsimonious Model RLM Results}
    \vspace{0.5em}
    {\scriptsize
    \begin{tabular}{lrrrrrr}
    \toprule
    \textbf{Variable} & \textbf{coef} & \textbf{std err} & \textbf{z} & \textbf{$P>|z|$} & \textbf{[0.025} & \textbf{0.975]} \\
    \midrule
    $\beta_0$ (const) & 58.47 & 0.943 & 62.02 & 0.000 & 56.63 & 60.32 \\
    $\beta_2$ ($x_2$: energy) & -6.18 & 1.038 & -5.95 & 0.000 & -8.22 & -4.15 \\
    $\beta_3$ ($x_3$: valence) & -7.32 & 1.055 & -6.94 & 0.000 & -9.39 & -5.25 \\
    $\beta_4$ ($x_4$: acousticness) & -24.57 & 1.121 & -21.92 & 0.000 & -26.77 & -22.38 \\
    $\beta_5$ ($x_5$: instrumentalness) & -6.83 & 0.645 & -10.59 & 0.000 & -8.10 & -5.57 \\
    tempo\_c\_sq & -0.001 & 0.000 & -7.38 & 0.000 & -0.001 & -0.001 \\
    duration\_s\_c & 0.009 & 0.002 & 4.44 & 0.000 & 0.005 & 0.013 \\
    duration\_s\_c\_sq & -1.75e-05 & 1.52e-06 & -11.52 & 0.000 & -2.05e-05 & -1.45e-05 \\
    acoustic\_valence\_interact & 8.43 & 2.046 & 4.12 & 0.000 & 4.42 & 12.44 \\
    energy\_valence\_interact &	4.9520 &	3.802	& 1.302	& 0.193 &	-2.501 &	12.404 \\
    \bottomrule
    \end{tabular}
    }
    \end{block}
\end{frame}
%------------------------------------------------------------
\begin{frame}{Answering Our Research Questions: Part I}
    Our final model provides clear answers:

    \begin{description}
        \item[RQ1: The "Big Five"] \hfill \textbf{Confirmed.} \\
        \textit{Energy, valence, acousticness,} and \textit{instrumentalness} are all significant \textbf{negative} predictors. More produced (less acoustic) songs are strongly associated with higher popularity.
        
        \item[RQ2: The "Goldilocks Zone"] \hfill \textbf{Strongly Supported.} \\
        Significant negative coefficients on squared terms for \textit{tempo} and \textit{duration} confirm an inverted U-shape. Songs that are too slow/fast or too short/long are less popular.
    \end{description}
    
    \vspace{1.5em}
    \begin{beamercolorbox}[rounded=true,shadow=true]{block body}
    \textbf{Key Insight:} The most popular songs are highly produced with moderate tempo and duration.
    \end{beamercolorbox}
\end{frame}
%------------------------------------------------------------
\begin{frame}{Answering Our Research Questions: Part II}
    Continuing our analysis of interaction effects:

    \begin{description}
        \item[RQ3: The "Sad Banger"] \hfill \textbf{Not Supported.} \\
        The interaction term \textit{energy} $\times$ \textit{valence} was not statistically significant and was removed from the final model.
        
        \item[RQ4: The "Acoustic Amplification"] \hfill \textbf{Supported.} \\
        The interaction term \textit{acousticness} $\times$ \textit{valence} is significant and positive. For highly acoustic tracks, a song's emotional tone has a much weaker negative impact on its popularity.
    \end{description}
    
    \vspace{1.5em}
    \begin{alertblock}{Summary}
    \textbf{3 out of 4 hypotheses} were supported, revealing that song popularity follows predictable patterns with clear "sweet spots" and interaction effects.
    \end{alertblock}
\end{frame}
%------------------------------------------------------------
\begin{frame}{Visual Confirmation: Partial Regression Plots}
    The grid provides powerful visual confirmation of our final model's findings. Each subplot displays the relationship between \texttt{popularity} and a single predictor, after controlling for all other variables.
    
    \begin{figure}
        \includegraphics[width=0.7\textwidth]{plot_partial_regression_grid.png}
        \caption{Partial regression plots showing isolated effects of each predictor}
    \end{figure}
\end{frame}
%------------------------------------------------------------
\begin{frame}{Interpreting Partial Plots: Linear Effects}
    \begin{block}{Core Audio Features}
    Each shows a clear \textbf{downward-sloping blue line}.
    \end{block}
    
    \begin{alertblock}{Key Insight: Negative Relationships}
    Visually confirms negative linear relationships:
    \begin{itemize}
        \item \textbf{acousticness}: More acoustic $\rightarrow$ less popular
        \item \textbf{energy}: Higher energy $\rightarrow$ less popular
        \item \textbf{valence}: Happier songs $\rightarrow$ less popular
        \item \textbf{instrumentalness}: More instrumental $\rightarrow$ less popular
    \end{itemize}
    \end{alertblock}
    
    \begin{block}{Statistical Validation}
    Downward slopes confirm negative drivers of popularity.
    \end{block}
\end{frame}
%------------------------------------------------------------
\begin{frame}{Interpreting Partial Plots: Goldilocks Effects}
    \begin{block}{Tempo Squared Term}
    Clear \textbf{negative slope} - visual proof of inverted U-shape.
    \end{block}
    
    \begin{alertblock}{Tempo "Goldilocks Zone"}
    \begin{itemize}
        \item X-axis: Distance from average tempo, squared
        \item Extreme tempos $\rightarrow$ lower popularity
        \item \textbf{Confirms}: Moderate tempos preferred
    \end{itemize}
    \end{alertblock}
    
    \begin{block}{Duration Effects}
    \begin{itemize}
        \item \textbf{Linear}: Slightly positive (minor preference for longer)
        \item \textbf{Squared}: Steeply negative (penalty for very long)
    \end{itemize}
    \end{block}
    
    \begin{alertblock}{Duration Goldilocks}
    Initial gain overwhelmed by strong quadratic penalty.
    \end{alertblock}
\end{frame}
%------------------------------------------------------------
\begin{frame}{Interpreting Partial Plots: Interaction Effect}
    \begin{block}{Acoustic×Valence Interaction}
    Shows \textbf{positive slope} confirming significant interaction.
    \end{block}
    
    \begin{alertblock}{"Acoustic Amplification" Confirmed}
    \begin{itemize}
        \item Positive trend: acousticness×valence has distinct influence
        \item For acoustic tracks, emotional tone matters less
        \item Goes beyond individual variable effects
    \end{itemize}
    \end{alertblock}
    
    \begin{block}{Visual Validation Summary}
    Partial plots provide strong evidence for:
    \begin{itemize}
        \item Clear negative linear trends
        \item Powerful "Goldilocks" effects
        \item Significant positive interaction
        \item Isolated variable effects
    \end{itemize}
    \end{block}
\end{frame}
%------------------------------------------------------------
\begin{frame}{Conclusion}

    \begin{block}{The "Hit Song Formula" (According to our model)}
        The most popular songs tend to be:
        \begin{itemize}
            \item \textbf{Highly produced} (low \textit{acousticness}, low \textit{instrumentalness}).
            \item Of \textbf{moderate tempo and duration}—avoiding the extremes.
            \item Surprisingly, they lean towards lower \textit{energy} and lower \textit{valence} (less "happy").
            \item For acoustic songs, the emotional tone matters less for popularity.
        \end{itemize}
    \end{block}

    \begin{alertblock}{Primary Methodological Takeaway}
        This project is a case study in the importance of rigorous, iterative model diagnostics. Identifying the \textbf{root cause} of a diagnostic failure (outliers vs. skewness) is critical for choosing the correct remedy and ultimately producing a valid and defensible final model.
    \end{alertblock}

\end{frame}
%------------------------------------------------------------
\begin{frame}
    \Huge{\centerline{Thank You}}
    \Large{\centerline{\vspace{1cm}Questions?}}
\end{frame}
%------------------------------------------------------------
\end{document}
